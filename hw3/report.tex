\documentclass[12pt,a4paper]{article}
% Change "article" to "report" to get rid of page number on title page
\usepackage{amsmath,mathtools,amsfonts,amsthm,amssymb}
\usepackage{setspace}
\usepackage{Tabbing}
\usepackage{fancyhdr}
\usepackage{lastpage}
\usepackage{extramarks}
\usepackage{chngpage}
\usepackage{fourier}
\usepackage{soul,color}
\usepackage[usenames,dvipsnames]{xcolor}
\usepackage{graphicx,float,wrapfig}
\usepackage[utf8]{inputenc}
\usepackage{sidecap}
\usepackage{marvosym}
\usepackage{tikz, tikz-qtree}
\usepackage{tabularx, multirow}
\usepackage{enumerate}
\usepackage{hyperref}
\definecolor{gray99}{gray}{.99}
\usepackage{listings}
\usepackage[english]{babel}
\usepackage{placeins}
\usepackage{tikz}
\usepackage{tikz-qtree}
\usepackage{xspace}
\usepackage{mathtools}
\usepackage{tabulary}
\lstset{
	language=R,
	backgroundcolor=\color{gray99},
	tabsize=3,
	frame=single,
	keywordstyle=\ttfamily\bfseries\color{RoyalBlue},
	commentstyle=\ttfamily\color{ForestGreen},
	stringstyle=\ttfamily\color{Gray},
	breaklines=true,
	showstringspaces=false,
	basicstyle=\footnotesize\ttfamily,
	emph={label},
	xleftmargin=22pt,
	framexleftmargin=22pt,
	framexrightmargin=0pt,
	framexbottommargin=4pt,
	numbers=left,
	stepnumber=1
}
\usepackage{caption}
\DeclareCaptionFont{black}{\color{black}}{\bfseries}
\DeclareCaptionFormat{listing}{\parbox{\textwidth}{\hspace{8pt}#1#2#3}}
\captionsetup[lstlisting]{format=listing,labelfont=black,textfont=black, singlelinecheck=false, margin=0pt, font={bf,footnotesize}}

% In case you need to adjust margins:
\topmargin=-0.45in      %
\evensidemargin=0in     %
\oddsidemargin=0in      %
\textwidth=6.5in        %
\textheight=9.5in       %
\headsep=0.25in         %

% Special font
\newcommand{\cps}[2]{\ensuremath{[[{#1}]]_{\textstyle #2}}}

% Homework Specific Information
\newcommand{\hmwkTopic}{Predicting Crime Rates}
\newcommand{\hmwkTitle}{HW2 - \hmwkTopic}
\newcommand{\hmwkDueDate}{April 8, 2014}
\newcommand{\hmwkClass}{CS 199}
\newcommand{\hmwkAuthorNameA}{Sam Laane}
\newcommand{\hmwkAuthorEmailA}{laane2@illinois.edu}
\newcommand{\hmwkAuthorNameB}{José Vicente Ruiz}
\newcommand{\hmwkAuthorEmailB}{ruizcep2@illinois.edu}

% Setup the header and footer
\pagestyle{fancy}                                                       %
\lhead{\hmwkAuthorNameA \xspace \& \hmwkAuthorNameB}                                                 %
\chead{\hmwkClass}  %
\rhead{\hmwkTopic}     
                                                %
\lfoot{}                                                      %
\cfoot{\thepage}                                                        %
\rfoot{}                          %
\renewcommand\headrulewidth{0.4pt}                                      %
\renewcommand\footrulewidth{0.4pt}                                      %


%%%%%%%%%%%%%%%%%%%%%%%%%%%%%%%%%%%%%%%%%%%%%%%%%%%%%%%%%%%%%
% Make title
\title{\vspace{2in}\textmd{\hmwkClass\\\textbf{\hmwkTitle}}\\\normalsize\vspace{0.1in}\small{\hmwkDueDate}\\\vspace{4in}}
\date{}
\author{\textbf{\hmwkAuthorNameA} $\;$<\texttt{\href{mailto:laane2@illinois.edu}{\hmwkAuthorEmailA}}>\\\textbf{\hmwkAuthorNameB} $\;$<\texttt{\href{mailto:ruizcep2@illinois.edu}{\hmwkAuthorEmailB}}>}
%%%%%%%%%%%%%%%%%%%%%%%%%%%%%%%%%%%%%%%%%%%%%%%%%%%%%%%%%%%%%

\begin{document}
\begin{singlespace}

\begin{titlepage}
\maketitle
\thispagestyle{empty}
\end{titlepage}

% Uncomment the \tableofcontents and \newpage lines to get a Contents page
% Uncomment the \setcounter line as well if you do NOT want subsections
%       listed in Contents
%\setcounter{tocdepth}{1}
\tableofcontents
\newpage

% When problems are long, it may be desirable to put a \newpage or a
% \clearpage before each homeworkProblem environment

\clearpage

\section{Removing variables with missing values}
\subsection{Implementation}
\lstinputlisting[firstline=13, lastline=23]{crimes.R}

\section{Basics - Linear regression}
\subsection{Implementation}
\lstinputlisting[firstline=31, lastline=78]{crimes.R}

\subsection{Results}
 - Mean-squared error on the whole data: 1.66e-02 \\
 - Mean-squared error on the test data (20\%): 1.88e-02

\newpage
\subsubsection{Standard data}
\vspace{-0.5cm}
\begin{figure}[h!]
    \centering
    \includegraphics[width=0.7\textwidth,trim= 0 0 20 30, clip]{Linear_regression_residuals.pdf}
\end{figure}
\FloatBarrier

\vspace{-0.5cm}
\begin{figure}[h!]
    \centering
    \includegraphics[width=0.7\textwidth,trim= 0 0 20 30, clip]{Linear_regression_cook.pdf}
\end{figure}
\FloatBarrier

\newpage
\subsubsection{Box-Cox transformed data}
\vspace{-0.5cm}
\begin{figure}[h!]
    \centering
    \includegraphics[width=0.7\textwidth,trim= 0 0 20 30, clip]{Boxcox_regression_residuals.pdf}
\end{figure}
\FloatBarrier

\vspace{-0.5cm}
\begin{figure}[h!]
    \centering
    \includegraphics[width=0.7\textwidth,trim= 0 0 20 30, clip]{Boxcox_regression_cook.pdf}
\end{figure}
\FloatBarrier

\subsection{Conclusions}

\section{Basics - Nearest Neighbour regression}
\subsection{Implementation}
\lstinputlisting[firstline=85, lastline=108]{crimes.R}

\subsection{Results}
 - Mean-squared error on the test data (20\%): 3.64e-02


\vspace{-0.5cm}
\begin{figure}[h!]
    \centering
    \includegraphics[width=0.7\textwidth,trim= 0 0 20 30, clip]{NN_regression_residuals.pdf}
\end{figure}
\FloatBarrier

\subsection{Conclusions}

\section{Dealing with missing values}
\subsection{Implementation}
\lstinputlisting[firstline=110, lastline=159]{crimes.R}

\subsection{Results}
Linear Regression (imputed missing values):
 - Mean-squared error on the test data (20\%): 1.88e-02
Nearest Neighbours (imputed missing values):
 - Mean-squared error on the test data (20\%): 1.00e-01

\newpage
\subsubsection{Linear regression}
\vspace{-0.5cm}
\begin{figure}[h!]
    \centering
    \includegraphics[width=0.7\textwidth,trim= 0 0 20 30, clip]{Unk_linear_regression_residuals.pdf}
\end{figure}
\FloatBarrier

\vspace{-0.5cm}
\begin{figure}[h!]
    \centering
    \includegraphics[width=0.7\textwidth,trim= 0 0 20 30, clip]{Unk_linear_regression_cook.pdf}
\end{figure}
\FloatBarrier

\newpage
\subsubsection{Nearest Neighbours}
\vspace{-0.5cm}
\begin{figure}[h!]
    \centering
    \includegraphics[width=0.7\textwidth,trim= 0 0 20 30, clip]{Unk_NN_regression_residuals.pdf}
\end{figure}
\FloatBarrier

\subsection{Conclusions}

\section{Modified Nearest Neighbours}
\subsection{Implementation}
\lstinputlisting{nn-modified.R}

\vspace{-0.4cm}
\subsection{Conclusions}
Two different implementations have been tried for this exercise, one that computes the distance between two vectors with some question mark elements in a linear way (\texttt{distance\_lin}) and other that does the same but with vectorial operations (\texttt{distance\_vec}). \\

Unfortunately, the linear code that uses this distance functions and other \texttt{R} specific instructions to apply the function to matrix columns, like \texttt{outer}, were so slow during the execution that it was impossible to obtain results.

\end{singlespace}
\end{document}
